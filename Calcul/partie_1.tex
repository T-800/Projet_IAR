\documentclass[a4paper,12pt]{report}

\usepackage[francais]{babel}
\usepackage[T1]{fontenc}
\usepackage[latin1,utf8]{inputenc}
\usepackage{soul}
\usepackage{lmodern}
\usepackage{amsmath}
\usepackage{amssymb}
\usepackage{mathrsfs}
\usepackage{amsmath}
\usepackage{amsfonts}
\usepackage{amssymb}
\usepackage{graphics}
\usepackage{pgf,tikz}
\usetikzlibrary{quotes,angles}
\usetikzlibrary{arrows}
\usepackage{multirow}
\usepackage{listings}
\usepackage{algorithm}
\usepackage{algorithmic}
\usepackage{graphicx}
\setlength{\parindent}{0cm}
\setlength{\parskip}{1ex plus 0.5ex minus 0.2ex}
\newcommand{\hsp}{\hspace{20pt}}
\newcommand{\HRule}{\rule{\linewidth}{0.5mm}}

\begin{document}

\subsubsection{.}
\label{subs:}

Nous avons $\mu(\overrightarrow{P}) = \tau$ qui est le moment cinétique de la force $P$. Par rapport au point $O$.

\begin{tikzpicture}[
    axis/.style={very thick, ->, >=stealth'},
    important line/.style={thick},
    dashed line/.style={dashed, thin},
    pile/.style={thick, ->, >=stealth', shorten <=2pt, shorten
    >=2pt},
    every node/.style={color=black}]

  \draw[axis] (-0.1,0)  -- (1.1,0) node(xline)[right]{$x$};
  \draw[axis] (0,-0.1) -- (0,1.1) node(yline)[above] {$y$};

   \filldraw[fill=white,line width=1pt]
    (2,0)circle(.12cm);
  \draw[line width=.6pt]
    (2,0) +(-135:.12cm) -- +(45:.12cm)
            +(-45:.12cm) -- +(135:.12cm);
  \draw

    (1,2.5) coordinate (m) node[left] {M}
    -- (3,5) coordinate (o) node[right] {O}
    -- (3,3) coordinate (p) node[right]{P}

    pic["$\theta$", draw=orange, <->, angle eccentricity=1.2, angle radius=1cm]
   {angle=m--o--p};
   \draw (1,2.5) node[black]{$\bullet$};
   \draw (3,5) node[black]{$\bullet$};

   \draw[thick,->]
   (3,5) coordinate (O)  node[right]{}
   -- (3,3) coordinate (p) node[right]{P};

\end{tikzpicture}


Nous avons ici $x > o$ et $\mu(\overrightarrow{P}) < 0 \Rightarrow \mu(\overrightarrow{P}) = - |\mu(\overrightarrow{P})|$.\\
$$
|\mu(\overrightarrow{P})| = OM.P.\sin{\theta} = OM.P. \frac{x}{OM} = Px
$$

Ce qui nous donne $\mu(\overrightarrow{P}) = -Px$\\



\begin{tikzpicture}[
    axis/.style={very thick, ->, >=stealth'},
    important line/.style={thick},
    dashed line/.style={dashed, thin},
    pile/.style={thick, ->, >=stealth', shorten <=2pt, shorten
    >=2pt},
    every node/.style={color=black}]

  \draw[axis] (-0.1,0)  -- (1.1,0) node(xline)[right]{$x$};
  \draw[axis] (0,-0.1) -- (0,1.1) node(yline)[above] {$y$};

   \filldraw[fill=white,line width=1pt]
    (2,0)circle(.12cm);
    \filldraw[fill=black,line width=1pt]
     (2,0)circle(.03cm);
  \draw

    (6,2.5) coordinate (m) node[left] {M}
    -- (3,5) coordinate (o) node[right] {O}
    -- (3,3) coordinate (p) node[right]{P}

    pic["$\theta$", draw=orange, <->, angle eccentricity=1.2, angle radius=1cm]
   {angle=p--o--m};
   \draw (6,2.5) node[black]{$\bullet$};
   \draw (3,5) node[black]{$\bullet$};

   \draw[thick,->]
   (3,5) coordinate (O)  node[right]{}
   -- (3,3) coordinate (p) node[right]{P};

\end{tikzpicture}

Avec $x < 0$ et $\mu(\overrightarrow{P}) > 0 \Rightarrow \mu(\overrightarrow{P}) = |\mu(\overrightarrow{P})|$

$$
\mu(\overrightarrow{P}) = |\mu(\overrightarrow{P})| = OM.P.\sin{\theta}
$$

Comme $x < 0$ et que $\mu(\overrightarrow{P})$ doit être $> 0$ on a $\mu(\overrightarrow{P}) = -P.x$\\\\



\subsubsection{.}
\label{subs:}
\textbf{D'après le Théorem du moment cinétique nous avons} :

$$\dot{L} = \sum{}{\mu(\overrightarrow{F}_{ext})} $$

Dans le cadre de notre  projet nous supposons que la seul force extérieur est la force de pesanteur.
$\dot{L}$ est donc égal à:

$$\dot{L} = -m_1.g.x_1 - m_2.g.x_2 = - (m_1 + m_2).g.\frac{m_1.x_1 + m_2.x_2}{m_1 + m_2}$$

Avec $\frac{m_1.x_1 + m_2.x_2}{m_1 + m_2} = X_G$ qui représente le centre de masse du corps

On retrouve donc la formule :

$$\dot{L} = - (m_1 + m_2)g X_G$$

$m_i$ et $g$ étant des constantes nous avons bien
$$\dot{L} = 0 \Leftrightarrow X_G = 0$$

Et

$\ddot{L} = - (m_1 + m_2)g \dot{X}_G$



\subsubsection{.}
\label{subs:}

$$\overrightarrow{L} = I.\overrightarrow{\omega}$$
Avec : $I = \frac{1}{12}.m_i.l_i$ et $\omega = \dot{q}$

$L$ s'écrit donc
$$L = \frac{1}{12}.m_1.l_1.\dot{q}_1 + \frac{1}{12}.m_2.l_2.\dot{q}_2$$

$m_i$, $l_i$ étant des constantes $\dot{L} = 0 \Leftrightarrow \dot{q}_1 = 0$ et $\dot{q}_2 = 0$


\subsubsection{.}
\label{subs:}

$$X_g = \frac{m_1.x_1 + m_2.x_2}{m_1 + m_2}$$

Avec $x_1 = l_{c1} \cos{q_1}$ et $x_2 = l_{1} \cos{q_1} + l_{c2} \cos{q_1 + q_2}$

\textbf{SCHEMA}\\\\

$$X_G = \frac{m_1.l_{c1} \cos{q_1} + m_2.(l_{1} \cos{q_1} + l_{c2} \cos{q_1 + q_2})}{m_1 + m_2}$$

d'où
$$
\dot{x_1} = - \dot{q_1}.l_{c1}.\sin(q_1)
$$
$$
\dot{x_2} = - \dot{q_1}.l_{1}.\sin(q_1) - (\dot{q_1} + \dot{q_2}).l_{c2}.\sin(q_1 + q_2)
$$

$\dot{X}_G$ se ré-écrit :
$$\dot{X}_G = \frac{m_1}{m_1 + m_2}.(- \dot{q_1}.l_{c1}.\sin(q_1)) + \frac{m_2}{m_1 + m_2}.(- \dot{q_1}.l_{1}.\sin(q_1) - (\dot{q_1} + \dot{q_2}).l_{c2}.\sin(q_1 + q_2))$$

On trouve que
$$\dot{q_1} = \dot{q_2} = 0 \Leftrightarrow \dot{X}_G = 0$$

$$
\dot{L} = - (m_1 + m_2)g X_G \Rightarrow \dot{L} = 0 \Leftrightarrow X_G = 0
$$

$$
\ddot{L} = - (m_1 + m_2)g \dot{X}_G \Rightarrow \ddot{L} = 0 \Leftrightarrow \dot{X}_G = 0 \Leftrightarrow \dot{q}_1 = 0 =\dot{q}_2 = 0
$$


d'où : \\
$q_1 = q_2 = 0$ et $X_G = 0 \Leftrightarrow L = \dot{L} = \ddot{L} = 0$
\end{document}
