\documentclass[a4paper,12pt]{report}

\usepackage[francais]{babel}
\usepackage[T1]{fontenc}
\usepackage[latin1,utf8]{inputenc}
\usepackage{soul}
\usepackage{lmodern}
\usepackage{amsmath}
\usepackage{amssymb}
\usepackage{mathrsfs}
\usepackage{amsmath}
\usepackage{amsfonts}
\usepackage{amssymb}
\usepackage{graphics}
\usepackage{pgf,tikz}
\usepackage{multirow}
\usepackage{listings}
\usepackage{algorithm}
\usepackage{algorithmic}
\usepackage{graphicx}
\usetikzlibrary{quotes,angles}
\usetikzlibrary{arrows}
\setlength{\parindent}{0cm}
\setlength{\parskip}{1ex plus 0.5ex minus 0.2ex}
\newcommand{\hsp}{\hspace{20pt}}
\newcommand{\HRule}{\rule{\linewidth}{0.5mm}}


\begin{document}


\section*{Moment cinétique}
\label{subs:}

\subsection*{Moment d'une force}
Soit $\mu(\overrightarrow{P})$ le moment cinétique de la force $P$, appliquée à un corps M, par rapport au point $O$.
$$\overrightarrow{\mu}(\overrightarrow{P}) = \overrightarrow{OM} \wedge \overrightarrow{P}$$
Soit $\mu(\overrightarrow{P})$ la mesure algébrique de ce moment. On peut distinguer deux cas:\\
\begin{itemize}
\item[Cas 1:] Le corps se trouve à droite du point de repère: ~~\\

\begin{center}
\begin{tikzpicture}[
    axis/.style={very thick, ->, >=stealth'},
    important line/.style={thick},
    dashed line/.style={dashed, thin},
    pile/.style={thick, ->, >=stealth', shorten <=2pt, shorten
    >=2pt},
    every node/.style={color=black}]

  \draw[axis] (-0.1,0) -- (1.1,0) node(xline)[right]{$x$};
  \draw[axis] (0,-0.1) -- (0,1.1) node(yline)[above] {$y$};

  \filldraw[fill=white,line width=1pt] (2.5,0)circle(.12cm);
  \filldraw[fill=black,line width=1pt] (2.5,0)circle(.03cm);
  \draw (2 ,0) node[black]{$z$};

  \draw % segment OMP
    (4,0) coordinate (m) node[left] {O}
    -- (6,3) coordinate (o) node[right] {M}
    -- (6,2) coordinate (p) node[right]{$\overrightarrow{P}$}
    pic["$\theta$", draw=orange, <->, angle eccentricity=1.2, angle radius=1cm] {angle=m--o--p}; % Angle theta

   \draw (4,0) node[black]{$\bullet$}; % Point O
   \draw (6,3) node[black]{$\bullet$}; % Point M
   \draw (6.9,2) node[black]{$\overrightarrow{\tau}$};
   \filldraw[fill=white,line width=1pt] (7.5,2)circle(.12cm);
   \draw[line width=.6pt]
     (7.5,2) +(-135:.12cm) -- +(45:.12cm)
             +(-45:.12cm) -- +(135:.12cm);


   \draw[thick,->] (6,3) -- (6,1); % vector P

\end{tikzpicture}
\end{center}
~~\\
Nous avons ici $x > o$ et $\mu(\overrightarrow{P}) < 0$,\ d'où \ \ $\Rightarrow \ \ \mu(\overrightarrow{P}) = - |\mu(\overrightarrow{P})|$, or:
$$
|\mu(\overrightarrow{P})| = OM.P.\sin{\theta} = OM.P. \frac{x}{OM} = Px
$$

Ce qui nous donne donc:
$$ \mu(\overrightarrow{P}) = -Px $$\\
\item[Cas 2:] Le corps se trouve à gauche du point de repère: ~~\\
\begin{center}

\begin{tikzpicture}[
    axis/.style={very thick, ->, >=stealth'},
    important line/.style={thick},
    dashed line/.style={dashed, thin},
    pile/.style={thick, ->, >=stealth', shorten <=2pt, shorten
    >=2pt},
    every node/.style={color=black}]

  \draw[axis] (-0.1,0) -- (1.1,0) node(xline)[right]{$x$};
  \draw[axis] (0,-0.1) -- (0,1.1) node(yline)[above] {$y$};

  \filldraw[fill=white,line width=1pt] (2.5,0)circle(.12cm);
  \filldraw[fill=black,line width=1pt] (2.5,0)circle(.03cm);
  \draw (2 ,0) node[black]{$z$};

  \draw % segment OMP
    (5,0) coordinate (m) node[left] {O}
    -- (3,3) coordinate (o) node[right] {M}
    -- (3,2) coordinate (p) node[left]{$\overrightarrow{P}$}
    pic["$\theta$", draw=orange, <->, angle eccentricity=1.2, angle radius=1cm] {angle=p--o--m};

   \draw (2 ,2) node {$\overrightarrow{\tau}$};
   \filldraw[fill=white,line width=1pt] (2,1.5)circle(.12cm);
   \filldraw[fill=black,line width=1pt] (2,1.5)circle(.03cm);

   \draw (5,0) node {$\bullet$}; % Point O
   \draw (3,3) node {$\bullet$}; % Point M

   \draw[thick,->] (3,3) -- (3,1);
\end{tikzpicture}

\end{center}
~~\\
Ici, $x < 0$ et $\mu(\overrightarrow{P}) > 0 \ \ \Rightarrow \ \ \mu(\overrightarrow{P}) = |\mu(\overrightarrow{P})| = OM.P.\sin{\theta}$;\\
Comme $x < 0$ alors:
$$\mu(\overrightarrow{P}) = -P.x$$\\

\end{itemize}
On a donc toujours:
$$\mu(\overrightarrow{P}) = -P.x$$

\subsection*{Moment cinétique et conditions d'équilibre}
\label{subs:}

Les conditions d'équilibre du pendule sont les suivantes:
$$ \dot{q_1} = \dot{q_1} = 0 \ \ \ \ et \ \ \ \ X_G = 0 $$
qui se traduisent par des vitesses angulaires nulles et l'alignement du centre de masse par rapport au point de contact.\\

Nous allons maintenant exprimer ces conditions en fonction du moment cinétique et de ces deux premières dérivées.

\subsubsection{Moment cinétique}
Les seules forces extérieures appliquées à l'IDP sont les forces de gravité appliquées aux deux corps:
$$ L = \overrightarrow I \overrightarrow{\dot{q}} $$
Avec : $$I_i = \frac{1}{12}.m_i.l_i$$
$L$ s'écrit donc
$$L = \frac{1}{12} m_1 l_1 \dot{q}_1 + \frac{1}{12} m_2 l_2 \dot{q}_2$$

$m_i$, $l_i$ étant des constantes, on en déduit:
$$ L = 0 \ \ \ \Leftrightarrow \ \ \dot{q}_1 = \dot{q}_2 = 0 $$

\subsubsection{Première dérivée du moment cinétique}


D'après le Théorem du moment cinétique:

$$\dot{L} = \sum{}{\mu(\overrightarrow{F}_{ext})} $$
$$\dot{L} = -m_1.g.x_1 - m_2.g.x_2 = - (m_1 + m_2).g.\frac{m_1.x_1 + m_2.x_2}{m_1 + m_2}$$
Avec $$\frac{m_1.x_1 + m_2.x_2}{m_1 + m_2} = X_G$$


On retrouve donc la formule :
$$\dot{L} = - (m_1 + m_2)g X_G$$

$m_i$ et $g$ étant des constantes nous avons bien
$$\dot{L} = 0 \ \ \ \Leftrightarrow \ \ \ X_G = 0$$

\subsubsection{Seconde dérivée du moment cinétique}
La seconde dérivée du moment cinétique s'écrit:
$$\ddot{L} = - (m_1 + m_2)g \dot{X}_G$$

$$X_G = \frac{m_1 x_1 + m_2 x_2}{m_1 + m_2}$$
~~\\
\begin{center}
\begin{tikzpicture}[
    axis/.style={very thick, ->, >=stealth'},
    important line/.style={thick},
    dashed line/.style={dashed, thin},
    pile/.style={thick, ->, >=stealth', shorten <=2pt, shorten
    >=2pt},
    every node/.style={color=black}]

  \draw[axis] (-0.1,0) -- (1.1,0) node(xline)[right]{$x$};
  \draw[axis] (0,-0.1) -- (0,1.1) node(yline)[above] {$y$};

  \filldraw[fill=white,line width=1pt] (2.5,0)circle(.12cm);
  \filldraw[fill=black,line width=1pt] (2.5,0)circle(.03cm);

  \draw (2 ,0) node{$z$};

  %Sol
  \draw[thick] (3,0) -- (7,0);
  \draw

    (5,0) coordinate (u)
    -- (4,0) coordinate (o) node[above left] {O}
    -- (5,1) coordinate (m)
    -- (2,3) coordinate (p)
    pic["$q_1$", draw=orange, <->, angle eccentricity=1.2, angle radius=1cm]
   {angle=u--o--m};

   \draw [dashed]
    (7,1) coordinate (w)
    -- (5,1) coordinate (v)
    -- (7,3) coordinate (z)
    pic["$q_1$", draw=orange, <->, angle eccentricity=1.2, angle radius=1cm]
   {angle=w--v--z};

   \draw [dashed]
    (7,3) coordinate (ww)
    -- (5,1) coordinate (vv)
    -- (4.9,1.1) coordinate (zz)
    pic["$q_2$", draw=orange, <->, angle eccentricity=1.2, angle radius=1cm]
   {angle=ww--vv--zz};


   \draw (4,0) node[black]{$\bullet$};
   \draw (4.5,0.5) node[black]{$\bullet$};
   \draw (5,1) node[black]{$\bullet$};
   \draw (3.5,2) node[black]{$\bullet$};
   \draw (2,3) node[black]{$\bullet$};
   \draw (4.15,0.5) node[black]{$l_{c1}$};
   \draw (4.15,1.5) node[black]{$l_{c2}$};


   \draw[thick,<->] (4,-0.5) -- (4.5,-0.5) ;
   \draw (4,-0.8) node[black]{$x_1 = l_{c1} \cos(q_1)$};

   \draw[thick,<->] (4,3.5) -- (5,3.5);
   \draw[thick,<->] (4,3.5) -- (3.5,3.5);
   \draw (4,3.8) node[black]{$x_2 = l_{1} \cos(q_1) + l_{c2} \cos(q_1 + q_2)$};

\end{tikzpicture}
\end{center}
~~\\
Avec,
$$ x_1 = l_{c1} \cos{(q_1)} $$
$$ x_2 = l_{1} \cos{(q_1)} + l_{c2} \cos{(q_1 + q_2)} $$

d'où
$$
\dot{x_1} = - \dot{q_1} l_{c1} \sin(q_1)
$$
$$
\dot{x_2} = - \dot{q_1} l_{1} \sin(q_1) - (\dot{q_1} + \dot{q_2}) l_{c2} \sin(q_1 + q_2)
$$

$\dot{X}_G$ s'écrit donc  :
$$\dot{X}_G = \frac{m_1}{m_1 + m_2} (- \dot{q_1} l_{c1} \sin(q_1)) + \frac{m_2}{m_1 + m_2} (- \dot{q_1} l_{1} \sin(q_1) - (\dot{q_1} + \dot{q_2}) l_{c2} \sin(q_1 + q_2))$$

Ce qui nous donne:
$$\dot{q_1} = \dot{q_2} = 0 \ \ \ \Leftrightarrow \ \ \ \dot{X}_G = 0$$

d'où
$$
 \ddot{L} = 0 \ \ \ \Leftrightarrow \ \ \ \dot{X}_G = 0 \ \ \ \Leftrightarrow \ \ \  \dot{q}_1 = 0 =\dot{q}_2 = 0
$$

Finalement, on obtient la règle:
$$\dot{q_1} = \dot{q_2} = 0 \ \ et  \ \ X_G = 0  \ \ \ \Leftrightarrow \ \ \ L = \dot{L} = \ddot{L} = 0$$


\section*{Analyse de stabilité et calcul des gains}
On considère les deux équations du mouvement utilisées pour simuler le balancement du pendule:
$$
\left\{
    \begin{array}{ll}
        d_{11}\ddot{q_1} + d_{12}\ddot{q_2} = -h_1 - \phi_1  \\
        d_{21}\ddot{q_1} + d_{22}\ddot{q_2} = \tau -h_2 - \phi_2  \\
    \end{array}
\right.
$$
Ces deux équations sont équivalentes à:
$$
    \begin{pmatrix}
    d_{11} & d_{12} \\
    d_{21} & d_{22}
    \end{pmatrix}
    \begin{pmatrix}
    \ddot{q_1} \\
    \ddot{q_2}
    \end{pmatrix}
    =
    \begin{pmatrix}
    -h_1 - \phi_1 \\
    \tau -h_2 - \phi_2
    \end{pmatrix}
$$

Pour obtenir un système de la forme: $M\dot{x}=N(x)$, on ajoute des lignes et colonnes à chaque membre du système tout en gardant sa validité:
$$
    \begin{pmatrix}
    0 & 0 & d_{11} & d_{12} \\
    0 & 0 & d_{21} & d_{22} \\
    1 & 0 & 0 & 0 \\
    0 & 1 & 0 & 0
    \end{pmatrix}
    \begin{pmatrix}
    \dot{q_1} \\
    \dot{q_2} \\
    \ddot{q_1} \\
    \ddot{q_2}
    \end{pmatrix}
    =
    \begin{pmatrix}
    -h_1 - \phi_1 \\
    \tau -h_2 - \phi_2 \\
    \dot{q_1} \\
    \dot{q_2}
    \end{pmatrix}
$$
Pour obtenir un système du type $ \dot{x} = h(x) $, nous allons inverser la matrice M via la formule:
$$ M^{-1} = (det(M))^{-1} com^t (M) $$
avec $$ det(M) = d_{11}d_{22} - d_{12}d_{21} $$
d'où
$$
    M^{-1}
    =
    \begin{pmatrix}
    0 & 0 & 1 & 0 \\
    0 & 0 & 0 & 1 \\
    \frac{d_{22}}{det(M)} & -\frac{d_{12}}{det(M)} & 0 & 0 \\
    -\frac{d_{21}}{det(M)} & \frac{d_{11}}{det(M)} & 0 & 0
    \end{pmatrix}
$$

Ce qui nous donne:
$$
    \dot{x}
    =
    \begin{pmatrix}
    \dot{q_1} \\
    \dot{q_2} \\
    \ddot{q_1} \\
    \ddot{q_2}
    \end{pmatrix}
    =
    \begin{pmatrix}
    0 & 0 & 1 & 0 \\
    0 & 0 & 0 & 1 \\
    \frac{d_{22}}{det(M)} & -\frac{d_{12}}{det(M)} & 0 & 0 \\
    -\frac{d_{21}}{det(M)} & \frac{d_{11}}{det(M)} & 0 & 0
    \end{pmatrix}
    .
    \begin{pmatrix}
    -h_1 - \phi_1 \\
    \tau -h_2 - \phi_2 \\
    \dot{q_1} \\
    \dot{q_2}
    \end{pmatrix}
$$

$$
    \Rightarrow
    \dot{x}
    =
    h(x)
    =
    \begin{pmatrix}
    \dot{q_1} \\
    \dot{q_2} \\
    \frac{d_{22}}{det(M)}(-h_1 - \phi_1) - \frac{d_{12}}{det(M)}(\tau -h_2 - \phi_2) \\
    -\frac{d_{21}}{det(M)}(-h_1 - \phi_1) + \frac{d_{11}}{det(M)} (\tau -h_2 - \phi_2)
    \end{pmatrix}
$$


Et enfin, afin d'obtenir le système linéaire de type $\dot{x}=Ax$, en calculant:
$$A_{ij} = \frac{\partial h_i(x)}{\partial{x_j}}|_{x=0} \ (x-0)$$

Nous avons donc besoin des calculs de dérivées partielles suivant:
$$
d_{11} : \left\{
    \begin{array}{ll}
	\frac{\partial d_{11}}{\partial q_1}\ |_{x=0} = \frac{\partial d_{11}}{\partial \dot{q_1}}\ |_{x=0} = \frac{\partial d_{11}}{\partial \dot{q_2}}\ |_{x=0} = 0 \\
	\ \\
	\frac{\partial d_{11}}{\partial q_2}\ |_{x=0} = -2m_2l_1l_{c_2}sin(q_2)\\
    \end{array}
\right.
$$

$$
d_{12} : \left\{
    \begin{array}{ll}
	\frac{\partial d_{12}}{\partial q_1}\ |_{x=0} = \frac{\partial d_{11}}{\partial \dot{q_1}}\ |_{x=0} = \frac{\partial d_{11}}{\partial \dot{q_2}}\ |_{x=0} = 0 \\
	\ \\
	\frac{\partial d_{12}}{\partial q_2}\ |_{x=0} = -m_2l_1l_{c_2}sin(q_2)\\
    \end{array}
\right.
$$

$$
d_{22} : \left\{
    \begin{array}{ll}
	\frac{\partial d_{22}}{\partial q_1}\ |_{x=0} = \frac{\partial d_{22}}{\partial q_1}\ |_{x=0} = \frac{\partial d_{22}}{\partial \dot{q_1}}\ |_{x=0} = \frac{\partial d_{22}}{\partial \dot{q_2}}\ |_{x=0} = 0 \\
    \end{array}
\right.
$$

$$
h_1 : \left\{
    \begin{array}{ll}
	\frac{\partial h_1}{\partial q_1}\ |_{x=0} = \frac{\partial h_1}{\partial q_1}\ |_{x=0} = \frac{\partial h_1}{\partial \dot{q_1}}\ |_{x=0} = \frac{\partial h_1}{\partial \dot{q_2}}\ |_{x=0} = 0 \\
    \end{array}
\right.
$$

$$
h_2 : \left\{
    \begin{array}{ll}
	\frac{\partial h_2}{\partial q_1}\ |_{x=0} = \frac{\partial h_2}{\partial q_1}\ |_{x=0} = \frac{\partial h_2}{\partial \dot{q_1}}\ |_{x=0} = \frac{\partial h_2}{\partial \dot{q_2}}\ |_{x=0} = 0 \\
    \end{array}
\right.
$$

$$
\phi_1 : \left\{
    \begin{array}{ll}
	\frac{\partial \phi_1}{\partial \dot{q_1}}\ |_{x=0} = \frac{\partial \phi_1}{\partial \dot{q_2}}\ |_{x=0} = 0 \\
	\ \\
	\frac{\partial \phi_1}{\partial q_1}\ |_{x=0} = -(m_1 l_{c_1} + m_2 l_1)g sin(q_1^d) - m_2 l_{c_2} g sin(q_1^d + q_2^d)  \\
	\ \\
	\frac{\partial \phi_1}{\partial \dot{q_2}}\ |_{x=0} = - m_2 l_{c_2} g sin(q_1^d + q_2^d) \\
    \end{array}
\right.
$$


$$
\phi_2 : \left\{
    \begin{array}{ll}
	\frac{\partial \phi_2}{\partial \dot{q_1}}\ |_{x=0} = \frac{\partial \phi_2}{\partial \dot{q_2}}\ |_{x=0} = 0 \\
	\ \\
	\frac{\partial \phi_2}{\partial q_1}\ |_{x=0} = \frac{\partial \phi_2}{\partial q_2}\ |_{x=0} = - m_2 l_{c_2} g sin(q_1^d + q_2^d)  \\
    \end{array}
\right.
$$


$$
\tau : \left\{
    \begin{array}{ll}
	\frac{\partial \tau}{\partial q_1}\ |_{x=0} = k_d g(m_1 l_{c_1} sin(q_1^d) + m_2 l_1 sin(q_1^d) + m_2 l_{c_2} sin(q_1^d + q_2^d)) - m_2 l_{c_2} g sin(q_1^d + q_2^d) \\
	\ \\
	\frac{\partial \tau}{\partial q_2}\ |_{x=0} = k_d g(m_2 l_{c_2} sin(q_1^d + q_2^d)) - m_2 l_{c_2} g sin(q_1^d + q_2^d) \\
	\ \\
	\frac{\partial \tau}{\partial \dot{q_1}}\ |_{x=0} = k_{dd} g(m_1 l_{c_1} sin(q_1^d) + m_2 l_1 sin(q_1^d) + m_2 l_{c_2} sin(q_1^d + q_2^d)) + \frac{1}{12} k_p m_1 l_1^2 \\
	\ \\
	\frac{\partial \tau}{\partial \dot{q_2}}\ |_{x=0} = k_{dd} g(m_2 l_{c_2} sin(q_1^d + q_2^d)) + \frac{1}{12} k_p m_2 l_2^2 \\

    \end{array}
\right.
$$

d'où
$$
    A
    =
    \begin{pmatrix}
    0 & 0 & 1 & 0 \\
    0 & 0 & 0 & 1 \\
    A_{31} & A_{32} & A_{33} & A_{34} \\
    A_{41} & A_{42} & A_{43} & A_{44}
    \end{pmatrix}
$$

Avec,
$$
\left\{
    \begin{array}{ll}
	A_{31} = - [ (\frac{d_{22}}{det(M)} \frac{\partial \phi_1}{\partial{q_1}}) + (\frac{d_{12}}{det(M)} \frac{\partial (\tau - \phi_2)}{\partial{q_1}}) ] \ |_{x=0} \\

	\ \\

	A_{32} = - [ (\frac{d_{22}}{det(M)} \frac{\partial \phi_1}{\partial{q_2}}) + (\frac{\partial}{\partial q_2}\frac{d_{12}}{det(M)})(\tau - \phi_2) + (\frac{d_{12}}{det(M)} \frac{\partial (\tau - \phi_2)}{\partial{q_2}}) ] \ |_{x=0} \\

        \ \\

        A_{33} = - [ (\frac{d_{12}}{det(M)} \frac{\partial \tau}{\partial{\dot{q}_1}}) ] \ |_{x=0} \\

        \ \\

        A_{34} = - [ (\frac{d_{12}}{det(M)} \frac{\partial \tau}{\partial{\dot{q}_2}}) ] \ |_{x=0} \\

        \ \\

        A_{41} = [ (\frac{d_{21}}{det(M)} \frac{\partial \phi_1}{\partial{q_1}}) + (\frac{d_{11}}{det(M)} \frac{\partial (\tau - \phi_2)}{\partial{q_1}}) ] \ |_{x=0} \\

	\ \\

	A_{42} = [ \frac{\partial}{\partial q_2}(\frac{d_{21}}{det(M)})(h_1 + \phi_1) + \frac{d_{21}}{det(M)} \frac{\partial \phi_1}{\partial q_2} + (\frac{\partial}{\partial{q_2}} \frac{d_{11}}{det(M)}) (\tau - h_2 - \phi_2) + \frac{d_{11}}{det(M)} \frac{\partial (\tau - \phi_2)}{\partial{q_2}} ] \ |_{x=0} \\

        \ \\

        A_{43} = - [ (\frac{d_{11}}{det(M)} \frac{\partial \tau}{\partial{\dot{q}_1}}) ] \ |_{x=0} \\

        \ \\

        A_{44} = - [ (\frac{d_{11}}{det(M)} \frac{\partial \tau}{\partial{\dot{q}_2}}) ] \ |_{x=0} \\

    \end{array}
\right.
$$

On peut écrire le polynôme caractéristique de la matrice A de la manière suivante:

$$
    P_A(\lambda) = det(\lambda I_4 - A) =
    \left|
    \begin{pmatrix}
    \lambda & 0 & 1 & 0 \\
    0 & \lambda & 0 & 1 \\
    - A_{31} & - A_{32} & \lambda - A_{33} & - A_{34} \\
    - A_{41} & - A_{42} & - A_{43} & \lambda - A_{44}
    \end{pmatrix}
    \right|
$$
\end{document}
