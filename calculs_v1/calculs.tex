\documentclass[a4paper,12pt]{report}

\usepackage[francais]{babel}
\usepackage[T1]{fontenc}
\usepackage[latin1,utf8]{inputenc}
\usepackage{soul}
\usepackage{lmodern}
\usepackage{amsmath}
\usepackage{amssymb}
\usepackage{mathrsfs}
\usepackage{amsmath}
\usepackage{amsfonts}
\usepackage{amssymb}
\usepackage{graphics}
\usepackage{pgf,tikz}
\usepackage{multirow}
\usepackage{listings}
\usepackage{algorithm}
\usepackage{algorithmic}
\usepackage{graphicx}

\setlength{\parindent}{0cm}
\setlength{\parskip}{1ex plus 0.5ex minus 0.2ex}
\newcommand{\hsp}{\hspace{20pt}}
\newcommand{\HRule}{\rule{\linewidth}{0.5mm}}

\begin{document}

\section*{Analyse de stabilité et calcul des gains}
On considère les deux équations du mouvement utilisées pour simuler le balancement du pendule:
$$
\left\{
    \begin{array}{ll}
        d_{11}\ddot{q_1} + d_{12}\ddot{q_2} = -h_1 - \phi_1  \\
        d_{21}\ddot{q_1} + d_{22}\ddot{q_2} = \tau -h_2 - \phi_2  \\
    \end{array}
\right.
$$
Ces deux équations sont équivalentes à:
$$
    \begin{pmatrix}
    d_{11} & d_{12} \\
    d_{21} & d_{22}
    \end{pmatrix}
    \begin{pmatrix}
    \ddot{q_1} \\
    \ddot{q_2}
    \end{pmatrix}
    =
    \begin{pmatrix}
    -h_1 - \phi_1 \\
    \tau -h_2 - \phi_2
    \end{pmatrix}
$$

Pour obtenir un système de la forme: $M\dot{x}=N(x)$, on ajoute des lignes et colonnes à chaque membre du système tout en gardant sa validité:
$$
    \begin{pmatrix}
    0 & 0 & d_{11} & d_{12} \\
    0 & 0 & d_{21} & d_{22} \\
    1 & 0 & 0 & 0 \\
    0 & 1 & 0 & 0
    \end{pmatrix}
    \begin{pmatrix}
    \dot{q_1} \\
    \dot{q_2} \\
    \ddot{q_1} \\
    \ddot{q_2}
    \end{pmatrix}
    =
    \begin{pmatrix}
    -h_1 - \phi_1 \\
    \tau -h_2 - \phi_2 \\
    \dot{q_1} \\
    \dot{q_2}
    \end{pmatrix}
$$
Pour obtenir un système du type $ \dot{x} = h(x) $, nous allons inverser la matrice M via la formule:
$$ M^{-1} = (det(M))^{-1} com^t (M) $$
avec $$ det(M) = d_{11}d_{22} - d_{12}d_{21} $$
d'où
$$
    M^{-1}
    =
    \begin{pmatrix}
    0 & 0 & 1 & 0 \\
    0 & 0 & 0 & 1 \\
    \frac{d_{22}}{det(M)} & -\frac{d_{12}}{det(M)} & 0 & 0 \\
    -\frac{d_{21}}{det(M)} & \frac{d_{11}}{det(M)} & 0 & 0
    \end{pmatrix}
$$

Ce qui nous donne:
$$
    \dot{x}
    =
    \begin{pmatrix}
    \dot{q_1} \\
    \dot{q_2} \\
    \ddot{q_1} \\
    \ddot{q_2}
    \end{pmatrix}
    =
    \begin{pmatrix}
    0 & 0 & 1 & 0 \\
    0 & 0 & 0 & 1 \\
    \frac{d_{22}}{det(M)} & -\frac{d_{12}}{det(M)} & 0 & 0 \\
    -\frac{d_{21}}{det(M)} & \frac{d_{11}}{det(M)} & 0 & 0
    \end{pmatrix}
    .
    \begin{pmatrix}
    -h_1 - \phi_1 \\
    \tau -h_2 - \phi_2 \\
    \dot{q_1} \\
    \dot{q_2}
    \end{pmatrix}
$$

$$
    \Rightarrow
    \dot{x}
    =
    h(x)
    =
    \begin{pmatrix}
    \dot{q_1} \\
    \dot{q_2} \\
    \frac{d_{22}}{det(M)}(-h_1 - \phi_1) - \frac{d_{12}}{det(M)}(\tau -h_2 - \phi_2) \\
    -\frac{d_{21}}{det(M)}(-h_1 - \phi_1) + \frac{d_{11}}{det(M)} (\tau -h_2 - \phi_2)     
    \end{pmatrix}
$$


Et enfin, afin d'obtenir le système linéaire, soit un système de type $\dot{x}=Ax$, on calcul:
$$A_{ij} = \frac{\partial h_i(x)}{\partial{x_j}}|_{x=0} \ (x-0)$$
d'où 
$$
    A
    =
    \begin{pmatrix}
    0 & 0 & 1 & 0 \\
    0 & 0 & 0 & 1 \\
    A_{31} & A_{32} & A_{33} & A_{34} \\
    A_{41} & A_{42} & A_{43} & A_{44} 
    \end{pmatrix}
$$

Avec,
$$
\left\{
    \begin{array}{ll}
	A_{31} = - [ (\frac{d_{22}}{det(M)} \frac{\partial \phi_1}{\partial{q_1}}) + (\frac{d_{12}}{det(M)} \frac{\partial (\tau - \phi_2)}{\partial{q_1}}) ] \ |_{x=0} \\
	
	\ \\
    
	A_{32} = - [ (\frac{d_{22}}{det(M)} \frac{\partial \phi_1}{\partial{q_2}}) + (\frac{\partial}{\partial q_2}\frac{d_{12}}{det(M)})(\tau - \phi_2) + (\frac{d_{12}}{det(M)} \frac{\partial (\tau - \phi_2)}{\partial{q_2}}) ] \ |_{x=0} \\
        
        \ \\
        
        A_{33} = - [ (\frac{d_{12}}{det(M)} \frac{\partial \tau}{\partial{\dot{q}_1}}) ] \ |_{x=0} \\
        
        \ \\
        
        A_{34} = - [ (\frac{d_{12}}{det(M)} \frac{\partial \tau}{\partial{\dot{q}_2}}) ] \ |_{x=0} \\
        
        \ \\
        
        A_{41} = [ (\frac{d_{21}}{det(M)} \frac{\partial \phi_1}{\partial{q_1}}) + (\frac{d_{11}}{det(M)} \frac{\partial (\tau - \phi_2)}{\partial{q_1}}) ] \ |_{x=0} \\
	
	\ \\
    
	A_{42} = [ \frac{\partial}{\partial q_2}(\frac{d_{21}}{det(M)})(h_1 + \phi_1) + \frac{d_{21}}{det(M)} \frac{\partial \phi_1}{\partial q_2} + (\frac{\partial}{\partial{q_2}} \frac{d_{11}}{det(M)}) (\tau - h_2 - \phi_2) + \frac{d_{11}}{det(M)} \frac{\partial (\tau - \phi_2)}{\partial{q_2}} ] \ |_{x=0} \\
	
        \ \\
        
        A_{43} = - [ (\frac{d_{11}}{det(M)} \frac{\partial \tau}{\partial{\dot{q}_1}}) ] \ |_{x=0} \\
        
        \ \\
        
        A_{44} = - [ (\frac{d_{11}}{det(M)} \frac{\partial \tau}{\partial{\dot{q}_2}}) ] \ |_{x=0} \\
        
    \end{array}
\right.
$$

et le calcul des dérivées partielles suivant:
$$
d_{11} : \left\{
    \begin{array}{ll}
	\frac{\partial d_{11}}{\partial q_1}\ |_{x=0} = \frac{\partial d_{11}}{\partial \dot{q_1}}\ |_{x=0} = \frac{\partial d_{11}}{\partial \dot{q_2}}\ |_{x=0} = 0 \\
	\ \\
	\frac{\partial d_{11}}{\partial q_2}\ |_{x=0} = -2m_2l_1l_{c_2}sin(q_2)\\
    \end{array}
\right.
$$

$$
d_{12} : \left\{
    \begin{array}{ll}
	\frac{\partial d_{12}}{\partial q_1}\ |_{x=0} = \frac{\partial d_{11}}{\partial \dot{q_1}}\ |_{x=0} = \frac{\partial d_{11}}{\partial \dot{q_2}}\ |_{x=0} = 0 \\
	\ \\
	\frac{\partial d_{12}}{\partial q_2}\ |_{x=0} = -m_2l_1l_{c_2}sin(q_2)\\
    \end{array}
\right.
$$

$$
d_{22} : \left\{
    \begin{array}{ll}
	\frac{\partial d_{22}}{\partial q_1}\ |_{x=0} = \frac{\partial d_{22}}{\partial q_1}\ |_{x=0} = \frac{\partial d_{22}}{\partial \dot{q_1}}\ |_{x=0} = \frac{\partial d_{22}}{\partial \dot{q_2}}\ |_{x=0} = 0 \\
    \end{array}
\right.
$$

$$
h_1 : \left\{
    \begin{array}{ll}
	\frac{\partial h_1}{\partial q_1}\ |_{x=0} = \frac{\partial h_1}{\partial q_1}\ |_{x=0} = \frac{\partial h_1}{\partial \dot{q_1}}\ |_{x=0} = \frac{\partial h_1}{\partial \dot{q_2}}\ |_{x=0} = 0 \\
    \end{array}
\right.
$$

$$
h_2 : \left\{
    \begin{array}{ll}
	\frac{\partial h_2}{\partial q_1}\ |_{x=0} = \frac{\partial h_2}{\partial q_1}\ |_{x=0} = \frac{\partial h_2}{\partial \dot{q_1}}\ |_{x=0} = \frac{\partial h_2}{\partial \dot{q_2}}\ |_{x=0} = 0 \\
    \end{array}
\right.
$$

$$
\phi_1 : \left\{
    \begin{array}{ll}
	\frac{\partial \phi_1}{\partial \dot{q_1}}\ |_{x=0} = \frac{\partial \phi_1}{\partial \dot{q_2}}\ |_{x=0} = 0 \\
	\ \\
	\frac{\partial \phi_1}{\partial q_1}\ |_{x=0} = -(m_1 l_{c_1} + m_2 l_1)g sin(q_1^d) - m_2 l_{c_2} g sin(q_1^d + q_2^d)  \\
	\ \\
	\frac{\partial \phi_1}{\partial \dot{q_2}}\ |_{x=0} = - m_2 l_{c_2} g sin(q_1^d + q_2^d) \\
    \end{array}
\right.
$$


$$
\phi_2 : \left\{
    \begin{array}{ll}
	\frac{\partial \phi_2}{\partial \dot{q_1}}\ |_{x=0} = \frac{\partial \phi_2}{\partial \dot{q_2}}\ |_{x=0} = 0 \\
	\ \\
	\frac{\partial \phi_2}{\partial q_1}\ |_{x=0} = \frac{\partial \phi_2}{\partial q_2}\ |_{x=0} = - m_2 l_{c_2} g sin(q_1^d + q_2^d)  \\
    \end{array}
\right.
$$

\section*{1. Formules}

$\label{eq:L}
	\dot{L} = -( m_{1} + m_{2})gX_{G} $


$	\ddot{L} = -( m_{1} + m_{2})g\dot{X}_{G} $


$	\tau = k_{dd}\ddot{L} + k_{d}\dot{L} + k_{p}L + \tau^{d} $


$	\tau^{d} = m_{2} l_{c2} g \cos(q_{1} + q_{2}) $

$	\tau = - k_{v}\dot{X}_{G} - k_{x}X_{G} + k_{p}L + \tau^{d} $

$	\dot{x} = f(x) + g(x) . u $

$	\dot{x} = Ax $

$	\lambda^{4} + (b_{1} k_{dd} - b_{2} k_{p} )\lambda^{3} + (b_{3} k_{d} - : \alpha)\lambda^{2} + (b_{4} k_{p} )\lambda + a = $

$	\lambda_{1} = \lambda_{2} = \lambda_{3} = \lambda_{4} = -p $

$	(\lambda + p)^{4} = \lambda^{4} + 4p\lambda^{3} + 6p^{2}\lambda^{2} +  4p^{3}\lambda : +  p^{4} = $



\section*{2. Explication des formules et variables}
\label{sec:Explication des formules et variables}


$	l_{1} $
Longueur de l'objet1, cet objet est attaché au sol par un point fixe.\\

$	l_{2} $
Longueur de l'objet2, cet objet est attaché à l'objet1 par une jointure à une de ses éxtrémités.\\

$	q_{1} $
Angle de l'objet 1 par rapport au sol, au niveau de la jointure.\\

$	q_{2} $
Angle de l'objet 2 par rapport au (sol ou de l'objet 1 ??), au niveau de la jointure.\\


$	m_{1} $
Masse de l'objet1 la masse est considédée uniforme.\\

$	m_{2} $
Masse de l'objet2 la masse est considédée uniforme.\\

$	l_{c1} $
Longueur entre une éxtrémité et le centre de masse (pas sûr) ($\Longleftrightarrow$ barycentre ?). Le point change en fonction de l'inclinaison de l'objet non ? - Et pour trouver le barycentre comme ça il faut utiliser une intégrale ?\\


$	l_{c2} $


$	I_{1} $
Moment d'inertie de l'objet1.\\


$	I_{2} $
Moment d'inertie de l'objet2.\\



\subsection*{2. Autres Variables}
\label{sub:Autres Variables}


$	L $

Moment angulaire par rapport a un point le point choisi est le point de contact au sol.\\


$	X_{G} = 0 $

Signifie que mon centre de masse "global" est aligné avec le point de contact au sol "$=0$".\\


$	L = 0 $

Comme L est exprimé en fonction du point de contact au sol avoir $	L = 0 $ signifie que l'on veut que toutes les forces s'annule ???.\\


$	\dot{X}_{G} = 0 $
.\\


$	\dot{q_{1}} = \dot{q_{2}} = 0 $

Dérivé de $q_{i}$ par rapport au temps on veut que $	\dot{q_{1}} = \dot{q_{2}} $ soit égal a $0$ signifie que nos angles ne bougent pas (on se déplace pas )\\



$	L = \dot{L} = \ddot{L} = 0 $
Comme on veut que $	X_{G} = 0 $ et  $	\dot{X}_{G} = 0 $ et que $ \dot{L}$ et $\ddot{L}$ sont exprimmés respectivement en fonction de $	X_{G} = 0 $ et  $	\dot{X}_{G} = 0 $ on a donc que $	L = \dot{L} = \ddot{L} = 0 $



$	k_{dd} $

$	k_{d} $

$	k_{p} $

$	\tau $
Je pense que $\tau$ représente notre configuration actuelle https://en.wikipedia.org/wiki/Torque.
Si $\tau^{d}$ est le moment de la force désiré alors on est equilibré si $\tau = \tau^{d}$. Car pour un équilibre on veut $	L = \dot{L} = \ddot{L} = 0 $.\\

$	\tau^{d} $

$	q_{1}^{d} $

$	q_{2}^{d} $

$	k_{v} = ( m_{1} + m_{2})g k_{dd} $

$	k_{x} = ( m_{1} + m_{2})g k_{d} $

$	x = (q_{1} - q_{1}^{d}, q_{2} - q_{2}^{d}, \dot{q}_{1}, \dot{q}_{2}) $

$	 u = \tau $

$	\dot{x} = h(x) $

$	A = \frac{\partial h}{\partial x} \mid x = 0 $
.\\
$	b_{i} $

$	a $
Accélération.\\

$	\alpha $
Angular acceleration.\\


\end{document}
